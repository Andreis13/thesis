\section{Domain Analysis}
\phantomsection

% User Interfaces
% Maybe a brief history
% Types of interactions / user interfaces
% Keyboard
% Mouse
% Gamepad
% Wii-like controllers
% Feedback
% All-in-one mobile device
% Native applications
% The Web context
% The problem of real-time over network (examples of usual contexts like sending notifications)
% WebSockets vs WebRTC (with examples of applications like Hangouts)
% Gaming (with examples from Google like the Starwars game and the Sports game)
% The need of a framework


\subsection{User Interfaces and Input Devices}

% Before diving in the deeps of real-time communication over the Web, it is necessary to review the path that people took in order to interact with computers in the first place.

At the very beginning of the computer era, the devices that we now know as PCs, laptops and tablets were immense pieces of machinery that occupied entire laboratories and only highly qualified individuals had access to them. These machines where developed in order to perform various numerical problems by executing digital computations. As the saying goes, a machine can only do what a man tells it to do, so engineers had to come up with different means to set tasks for these electronic beasts, that is, a need for \emph{user interfaces} arouse. Operators used large stacks of punched cards to feed instructions and data sets to computers. These punched cards in turn where created using specialized devices that also required some knowledge in the field. Over all it was a complex process that couldn't be easily grasped by an ordinary persons. At that time, however, the majority of computer user had PhDs and were trained to perform these very tasks so the difficulty of communicating with machines wasn't really a great problem.

% picture of ENIAC maybe

In time, the interest towards computers grew bigger among enthusiasts while the devices themselves where becoming smaller and smaller, eventually giving birth to the term of 'microcomputer'. Among the first microcomputers to get widespread popularity was the critically acclaimed Altair 8080 which represented a box with lights and switches on the front panel that where used to feed data and instructions into the computer and read the results back from it.

% picture of Altair 8080 front panel

As more hobbyists took a hold of such devices people discovered many uses for them beyond that of performing various mathematical operations. They learned how to connect teletypes to computers and this way the later got an interface that is familiar to all of us today, a keyboard. For some time people interacted with computers using command lines, but with the development of computer graphics and the creation of graphical user interfaces, a need for a new kind of controller appeared, specifically the need for a pointing device. Since then, a standard personal computer included a mouse and a keyboard among it's peripheral devices.

On the other hand, the development of input devices never stopped with the creation of the mouse and keyboard. A wealth of specialized devices was created to perform tasks that were specific to a particular use of the computer. One of the most notable fields that was moving the development of specialized peripherals was and still is the \emph{game development industry}. During the years, various types of controllers were developed, some of them are listed below:

\begin{description}
	\item [Gamepad]
	\item [Joystick]
	\item [Steering Wheel]
\end{description}

\subsection{Real-Time Communication over the Web} % about communication technologies

\subsubsection{Client - Server Communication} % Websockets

\subsubsection{Peer to Peer Communication} % WebRTC


\subsection{Real-Time Interaction over the Web} % game controllers + web

\subsubsection{Native Apps}

\subsubsection{Web Apps}

\subsection{Existing applications}

\subsubsection{Lightsaber Escape} % https://lightsaber.withgoogle.com/

\subsubsection{Super Sync Sports} % https://www.chrome.com/supersyncsports/



\clearpage
