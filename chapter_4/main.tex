\section{Economic Analysis}
\phantomsection

\subsection{Project Description}

The focus point of this thesis is the development of a toolkit that would
enable software engineers to create with ease web-based applications that
feature real-time communication with mobile devices and leverage all the
sensors and feedback elements that a modern smart-phone can deliver. In the
economic context, however, it is not enough to develop a framework, it is also
crucial to have a demonstration application that implements logic which uses
this framework, so that it can gain public attention and obtain necessary
financial investments to support its further development. The 'Snowfight' game
is a wonderful way to illustrate the power of real-time peer-to-peer
interaction between a desktop browser and a browser of a mobile device. It can
also effectively showcase the majority of the use-cases of the framework.

\subsection{Project Time Schedule}

In order to be successful, a project needs to have an effective development
methodology. Presently, \emph{Agile Software Development} gains popularity and
proves to be a flexible and robust way to create software. This methodology
implies an iterative and adaptive process of development and this in turn
influences the time table that has to be established.

\subsubsection{Objective Determination and SWOT Analysis}

The main objective of the project can be derived from its motivation. The
'Snowfight' should illustrate at its best the features of the toolkit
described in the thesis. The game should represent an isometric multiplayer
arcade in which participants can throw snowballs in each other and score
points. The players should control their characters via their smart-phones,
specifically through a web page rendered in the browser of the phone which
represents a some kind of game controller. The gameplay should feel very
responsive as if the player uses a wired gamepad.

Besides showing off technical features of the framework, the game should
appeal to potential players that would use it as an entertainment asset at
parties or team-building events. It should be engaging and well balanced as to
provide casual, yet challenging, and maybe even addictive, gameplay. It is
advisable for the controller part of the system be supported by most types of
mobile devices.

In order to get a better picture of the product and different paths of its
development it is a great idea to analyze the project from the perspective of
Strengths, Weaknesses, Opportunities and Threats, as it has proven to be an
effective technique in project management.

\paragraph{Strengths}

\begin{itemize}
    \item Intuitive user interface
    \item Multiplayer with up to 10 players
    \item Doesn't require any special programs to be installed
    \item Entertaining
\end{itemize}

\paragraph{Weaknesses}

\begin{itemize}
    \item The technology is available only on mid to high-end devices
    \item The game is somewhat bound by the limitations of the graphical framework
    \item Only one game mode
\end{itemize}

\paragraph{Opportunities}

\begin{itemize}
    \item Add several game modes
    \item Improve graphics art
    \item Use this project as a step-stone for another project
\end{itemize}

\paragraph{Threats}

\begin{itemize}
    \item Lack of visibility on the Internet
    \item Development of similar frameworks by other teams
\end{itemize}

\subsubsection{Time Schedule Establishment}

\subsection{Economic Motivation}

\subsubsection{Tangible and Intangible Asset Expenses}

\subsubsection{Salary Expenses}

\subsubsection{Individual Person Salary}

\subsubsection{Indirect Expenses}

\subsubsection{Wear and Deprecation}

\subsubsection{Product Cost}

\subsubsection{Economic Indicators and Results}

\subsection{Marketing Plan}

\subsection{First Clients}

\subsection{Economic Conclusions}






\clearpage
