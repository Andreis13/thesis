\section{Economic Analysis}

\subsection{Project Description}

The focus point of this thesis is the development of a toolkit that would
enable software engineers to create with ease web-based applications that
feature real-time communication with mobile devices and leverage all the
sensors and feedback elements that a modern smart-phone can deliver. In the
economic context, however, it is not enough to develop a framework, it is also
crucial to have a demonstration application that implements logic which uses
this framework, so that it can gain public attention and obtain necessary
financial investments to support its further development. The 'Snowfight' game
is a wonderful way to illustrate the power of real-time peer-to-peer
interaction between a desktop browser and a browser of a mobile device. It can
also effectively showcase the majority of the use-cases of the framework.

\subsection{Project Time Schedule}

In order to be successful, a project needs to have an effective development
methodology. Presently, \emph{Agile Software Development} gains popularity and
proves to be a flexible and robust way to create software. This methodology
implies an iterative and adaptive process of development and this in turn
influences the time table that has to be established.

\subsubsection{Objective Determination}

The main objective of the project can be derived from its motivation. The
'Snowfight' should illustrate at its best the features of the toolkit
described in the thesis. The game should represent an isometric multiplayer
arcade in which participants can throw snowballs in each other and score
points. The players should control their characters via their smart-phones,
specifically through a web page rendered in the browser of the phone which
represents a some kind of game controller. The gameplay should feel very
responsive as if the player uses a wired gamepad.

Besides showing off technical features of the framework, the game should
appeal to potential players that would use it as an entertainment asset at
parties or team-building events. It should be engaging and well balanced as to
provide casual, yet challenging, and maybe even addictive, gameplay. The
controller part of the system be supported by most types of mobile devices.

\newpage
\subsubsection{SWOT Analysis}

In order to get a better picture of the product and different paths of its
development it is a great idea to analyze the project from the perspective of
Strengths, Weaknesses, Opportunities and Threats, as it has proven to be an
effective technique in project management.

\paragraph{Strengths}

\begin{itemize}
    \item Intuitive user interface
    \item Multiplayer with up to 10 players
    \item Doesn't require any special programs to be installed
    \item Entertaining
\end{itemize}

\paragraph{Weaknesses}

\begin{itemize}
    \item The technology is available only on mid to high-end devices
    \item The game is somewhat bound by the limitations of the graphical framework
    \item Only one game mode
\end{itemize}

\paragraph{Opportunities}

\begin{itemize}
    \item Add several game modes
    \item Improve graphics art
    \item Use this project as a step-stone for another project
\end{itemize}

\paragraph{Threats}

\begin{itemize}
    \item Lack of visibility on the Internet
    \item Development of similar frameworks by other teams
\end{itemize}

\subsubsection{Time Schedule Establishment}

Most of IT projects' lifetimes consist of 5 basic steps: planning, research,
development, testing and deployment. These steps are in turn subdivided into
smaller parts in order to make the whole process manageable in terms of tasks.
Most of the time table will be allocated to the step of development and
testing as these are the steps with the most workload. Planning and research
by itself shouldn't take up too much time, as requirements usually evolve
during development, also research is a thing that usually never stops. The
development of the project can also be split in tasks that can be performed in
parallel thus minimizing the overall time. Total duration of the project is
computed using the equation \eqref{eq:duration}.

\begin{equation} \label{eq:duration}
D_T = D_F - D_S + T_R,
\end{equation}

\noindent where $D_T$ is the duration, $D_F$ -- the finish date, $D_S$ -- the
start date and $T_R$ -- reserve time. Table \ref{table:schedule} represents
the first iteration of the project schedule. The following notations are used
to improve readability: PM -- project manager, D -- developer, GD -- graphics
designer, SM -- sales manager.

\begin{table}[!h]
\begin{center}
\caption{Time schedule}
\renewcommand{\arraystretch}{1.5}
\begin{tabulary}{1\textwidth}{| C | C | C | C | C |}

\hline \textbf{Nr} & \textbf{Activity Name} & \textbf{Duration (days)} & \textbf{People involved} \\
\hline 1  & Project concept definition          & 5     & PM, GD, SM, D \\
\hline 2  & Market analysis                     & 10    & PM, SM        \\
\hline 3  & Domain analysis                     & 15    & PM, D         \\
\hline 4  & Product requirements specification  & 5     & PM, D         \\
\hline 5  & System modeling (UML)               & 10    & D             \\
\hline 6  & Graphic design                      & 10    & GD            \\
\hline 7  & Framework development               & 20    & D             \\
\hline 8  & Main game development               & 30    & PM, GD, D     \\
\hline 9  & Validation of results               & 10    & PM, GD, D, SM \\
\hline 10 & Documentation                       & 5     & D             \\
\hline 11 & Deployment and testing              & 10    & PM, D         \\
\hline 12 & Active marketing                    & 15    & SM            \\
\hline    & Total time to finish the system     & 145   &               \\
\hline

\end{tabulary}
\label{table:schedule}
\end{center}
\end{table}

\newpage
\subsection{Economic Motivation}

This section aims to give a perspective of the project from the economic point
of view. This includes the expenses and profits encountered as a result of the
project's activity as well as the various strategies of commercializing. All
the costs and prices are given in MDL (Moldavian lei) currency. Tangible and
intangible assets, indirect expenses will also be taken into account. One
important thing to mention is that the game is the fact that it is being
developed as an open-source project and it's primary goal is to popularize the
framework and toolkit underneath. Nevertheless, there are still opportunities
to commercialize the game, especially on the rising wave of the term
'Gamification' which gained substantial popularity lately in various
industries. Thus, the game in modified versions can be licensed to companies
that would like to 'gamify' their internal processes. Another monetization
model that might bring some profit and cover the development expenses is the
so called 'freemium' pricing model. This implies that the game should be
released for free with a limited set of features and the full package
of features would be unlocked to people that purchase a subscription of some
kind.


\newpage
\subsubsection{Asset Expense Evaluation}

In order to compute the potential cost of the product it is necessary to
evaluate the cost of various assets involved in the production of the project.
There are tangible and intangible assets that impose certain costs for the
project, the budget for both types of assets are exposed in tables
\ref{table:tangible_assets} and \ref{table:intangible_assets}.

\begin{table}[!h]
\begin{center}
\caption{Tangible asset expenses}
\renewcommand{\arraystretch}{1.5}
\begin{tabulary}{1\textwidth}{| C | C | C | C | C |}

\hline \textbf{Material} & \textbf{Specification}  & \textbf{Price per unit (MDL)} & \textbf{Quantity} & \textbf{Sum (MDL)}\\
\hline Workstation & Apple Macbook Pro 13' Retina   & 33000  & 2     & \multicolumn{1}{r|}{66000} \\
\hline Smartphone  & Nexus 5 and iPhone 5s          & 9000   & 2     & \multicolumn{1}{r|}{18000}  \\
\hline WebHosting  & Microsoft Azure VM             & 10000  & 1     & \multicolumn{1}{r|}{10000}  \\
\hline \multicolumn{4}{|r|}{Total}                                   & \multicolumn{1}{r|}{96000} \\
\hline
\end{tabulary}
\label{table:tangible_assets}
\end{center}
\vspace{-1.3em}
\end{table}

\begin{table}[!h]
\begin{center}
\caption{Intangible asset expenses}
\renewcommand{\arraystretch}{1.5}
\begin{tabulary}{1\textwidth}{| C | C | C | C | C |}

\hline
\textbf{Material} & \textbf{Specification} & \textbf{Price per unit (MDL)} & \textbf{Quantity} & \textbf{Sum (MDL)} \\
\hline License      & Enterprise Architect Desktop Edition License   & 1900  &   1   & \multicolumn{1}{r|}{1900} \\
\hline License      & Sublime Text 3 License                         & 1500  &   1   & \multicolumn{1}{r|}{1500} \\
\hline License      & Adobe Photoshop CC 2016                        & 2000  &   1   & \multicolumn{1}{r|}{2000} \\
\hline Domain Name  & Domain name registration fee                   & 500   &   1   & \multicolumn{1}{r|}{500} \\
\hline \multicolumn{4}{|r|}{Total}                                                   & \multicolumn{1}{r|}{4900} \\
\hline
\end{tabulary}
\label{table:intangible_assets}
\vspace{-1em}
\end{center}
\end{table}


\newpage
\subsubsection{Expendable Materials Expenses}

Besides the costs of assets, the development of the project implies some
direct expenses like consumable materials for the office. Even though the
amounts are  not that big, is important to keep track such expenses as they
tend to add up. These expenses are presented in the table
\ref{table:direct_expenses}.

\begin{table}[!h]
\begin{center}
\caption{Expendable materials expenses}
\renewcommand{\arraystretch}{1.5}
\begin{tabulary}{1\textwidth}{| C | C | C | C | C |}

\hline \textbf{Material} & \textbf{Specification} & \textbf{Price per unit (MDL)} & \textbf{Quantity} & \textbf{Sum (MDL)} \\
\hline Whiteboard      & Universal Dry Erase Board & 500    & 1     & \multicolumn{1}{r|}{500} \\
\hline Paper           & A4 - 500 sheet pack       & 60     & 2     & \multicolumn{1}{r|}{120} \\
\hline Marker          & Whiteboard marker         & 15     & 10    & \multicolumn{1}{r|}{150} \\
\hline Pen             & Blue pen                  & 5      & 10    & \multicolumn{1}{r|}{50} \\
\hline \multicolumn{4}{|r|}{Total}                                  & \multicolumn{1}{r|}{820} \\
\hline
\end{tabulary}
\label{table:direct_expenses}
\vspace{-1.5em}
\end{center}
\end{table}


\subsubsection{Salary Expenses}

It is obvious that a major part of project expenses go for the salaries of the
employees. This section describes the computations concerning salary funds.
The starting point is the distribution of per day salaries that the employees
are expected to receive. It looks like this: project manager - 400MDL, sales
manager - 300 MDL, developer - 380 MDL, graphics designer - 350 MDL.

\begin{table}[!ht]
\begin{center}
\caption{Salary expenses}
\renewcommand{\arraystretch}{2}
\begin{tabulary}{1\textwidth}{| C | C | C | C |}

\hline
\textbf{Employee} & \textbf{Work fund (days)} & \textbf{Salary per day (MDL)} & \textbf{Salary fund (MDL)} \\
\hline Project Manager      & 80    & 400   & \multicolumn{1}{r|}{32000} \\
\hline Developer            & 110   & 380   & \multicolumn{1}{r|}{41800} \\
\hline Graphics Designer    & 55    & 350   & \multicolumn{1}{r|}{19250} \\
\hline Sales Manager        & 40    & 300   & \multicolumn{1}{r|}{12000} \\
\hline
\multicolumn{3}{|r|}{Total}                 & \multicolumn{1}{r|}{105050} \\
\hline
\end{tabulary}
\label{table:salaries}
\end{center}
\end{table}

Given the data in the table \ref{table:salaries} it is necessary to compute the
amounts of money that should be payed to social services fund and medical
insurance fund. After this it is possible to sum everything up and obtain the
total work expense amount.

This year the social service fund is approved to be $23\%$, therefore the
salary expenses are computed according to the relation \eqref{eq:fs}.

\begin{equation}\label{eq:fs}
\begin{split}
 FS &= F_{re} \cdot T_{fs} \\
    &= 105050 \cdot 0.23 \\
    &= 24161.50,
\end{split}
\end{equation}

\noindent where $FS$ is the salary expense, $F_{re}$ is the salary expense
fund and $T_{fs}$ is the social service tax approved each year. The medical
insurance fund is computed as

\begin{equation}
\begin{split}
 MI &= F_{re} \cdot T_{mi}\\
    &= 105050 \cdot 0.045 \\
    &= 4727.25,
 \end{split}
\end{equation}

\noindent where $T_{mi}$ is the mandatory medical insurance tax approved each
year by law of medical insurance and this year it is $4.5\%$.

The total work expense fund can now be computed, given the amounts of social
service and medical insurance taxes.

\begin{equation}
\begin{split}
 WEF &= F_{re} + FS + MI\\
     &= 105050 + 24161.50 + 4727.25\\
     &= 133928.75,
\end{split}
\end{equation}

\noindent where $WEF$ is the work expense fund, $FS$ is the social fund and $MI$
is the medical insurance fund. In that way the total work expense fund was
computed.


\subsubsection{Individual Person Salary}

Along with total work expense fund, it is necessary to compute the annual
salary for the developer. Considering that the developer has a salary of 380
MDL per day and there are totally 250 working days in a year, so the gross
salary that the developer gets is:

\begin{equation}
GS = 380 \cdot 250 = 95000,
\end{equation}

\noindent where $GS$ is the gross salary computed in MDL.

Social fund and medical insurance taxes have the following values in MDL,
given the tax percentages approved this year:

\begin{equation}
\begin{split}
SF = 95000 \cdot 0.06 = 5700\\
MIF = 95000 \cdot 0.045 = 4725
\end{split}
\end{equation}

To compute the income tax it is necessary to now the amount of taxable salary:

\begin{equation}
\begin{split}
 TS &= GS - SF - MIF - PE \\
              &= 95000 - 5700 - 4725 - 10128\\
              &= 74447,
\end{split}
\end{equation}

\noindent where $TS$ is the taxable salary, $GS$ -- gross salary, $SF$ -- social
fund, $PE$ -- personal exemption, which this year is approved to be $10128$.

The last but not the least thing to be computed is the total income tax, which
is $7\%$ for income under 29640 MDL and $18\%$ for income over 29640 MDL:

\begin{equation}
\begin{split}
 IT &= TS - ST \\
      &= 29640 \cdot 0.07 + (74447 - 29640) \cdot 0.18 \\
      & = 2074.80 + 8065.30 = 10140.10,
 \end{split}
\end{equation}

\noindent where $IT$ is the income tax, $TS$ -- the taxed salary and $ST$ --
the salary tax. With all this now it is possible to find out the value of the
net income:

\begin{equation}
\begin{split}
 NS &= GS - IT - SF - MIF \\
            &= 95000 - 10140.10 - 5700 - 4725 \\
            &= 74434.90,
\end{split}
\end{equation}

\noindent where $NS$ is the net salary allocated for the employee, $GS$ -- gross
salary, $IT$ -- income tax, $SF$ -- social fund, $MIF$ -- medical insurance
fund.


\subsubsection{Indirect Expenses}

Besides direct expenses described in the previous sections there are such
things like electricity, internet traffic, water and other utilities that are
used in the office. These expenses are called indirect and the table
\ref{table:indirect_expenses} shows in what amounts they can impact project
costs.

\begin{table}[!ht]
\begin{center}
\caption{Indirect expenses}
\renewcommand{\arraystretch}{1.5}
\begin{tabulary}{1\textwidth}{| C | C | C | C | C |}

\hline \textbf{Material} & \textbf{Specification} & \textbf{Price per unit (MDL)} & \textbf{Quantity} & \textbf{Sum (MDL)} \\
\hline Internet     & Moldtelecom Subscription  & 200.00 per month  & 3     & \multicolumn{1}{r|}{600} \\
\hline Transport    & Public bus trip           & 3.00 per trip     & 110   & \multicolumn{1}{r|}{330} \\
\hline Electricity  & Union Fenosa              & 1.92 per kWh      & 250   & \multicolumn{1}{r|}{480} \\
\hline \multicolumn{4}{|r|}{Total}                                          & \multicolumn{1}{r|}{1410} \\
\hline
\end{tabulary}
\label{table:indirect_expenses}
\vspace{-2.5em}
\end{center}
\end{table}


\subsubsection{Wear and Depreciation}

Another important part of economic analysis is the computation of wear and
depreciation. It is a well known fact that any product decreases its value
with time. Depression will be computed uniformly for the whole project
duration, so that there are no accountancy issues. In other words, if a
product is planned for 3 years, it should be divided into 3 uniform parts
according to each year.

Straight line depreciation will be applied. Normally wear is computed
regarding to the type of asset. The notebook and single-board computer are
usable for a period of 3 years. Licenses will last for a single year. At first
tangible and intangible assets are summed up and then the salvage costs of
each of the items at the end of their period of use has to be subtracted:

\begin{equation}
 \begin{split}
  TAV &= \sum (AC - SV) \\
        &= (66000 - 46000) + (18000 - 9000) + (4900 - 0) \\
        &= 33900,
 \end{split}
\end{equation}

\noindent where $TAV$ is the total assets value, $AC$ -- assets cost, $SV$ --
salvage value. In order to get the yearly wear, divide total asset value by
the period of use of assets, being 3 years.

\begin{equation} \label{eq:wear}
 \begin{split}
  W_y &= TAV / T_{use} \\
                &= 33900/3\\
                &= 11300,
 \end{split}
\end{equation}

\noindent where $W_y$ is the wear per year, $TAV$ -- total assets value,
$T_{use}$ -- period of use. Relation \eqref{eq:wear} included tangible assets
which will last for 3 years and intangible assets which last only one year.
The initial value of assets in MDL was

\begin{equation}
 \begin{split}
  W &= W_y / D_y \cdot T_p\\
                   &= 11300 / 365  \cdot 135 \\
                   &= 4489,
 \end{split}
\end{equation}

\subsubsection{Product Cost}

With all the project expenses computed, it is easy to compute the product cost
which includes direct and indirect expenses, salary expenses and wear expenses
as shown in Table \ref{table:product_cost}.

\begin{table}[!ht]
\begin{center}
\caption{Total Product Cost}
\renewcommand{\arraystretch}{1.5}
\begin{tabulary}{1\textwidth}{| C | C | C |}

\hline \textbf{Expense type} & \textbf{Sum (MDL)} & \textbf{Percentage (\%)}\\
\hline Direct expenses              & 820     & 0.7    \\
\hline Indirect expenses            & 1410    & 1.2     \\
\hline Salary expenses              & 105050  & 90.04   \\
\hline Intangible asset expenses    & 4900    & 4.19     \\
\hline Asset wear expenses          & 4489    & 3.84    \\
\hline \textbf{Total product cost} & \textbf{116669} & \textbf{100}\\
\hline
\end{tabulary}
\label{table:product_cost}
\vspace{-2.5em}
\end{center}
\end{table}


\subsection{Economic Indicators and Results}

When it comes to selling the product, it is sometimes difficult to decide what
price to set for each sold copy or unit. In general there are two main
strategies that can be applied, and specifically: sell less with a high price
or sell more with a lower price. For the purpose of this thesis it is assumed
that the expected profit is $15\%$ of the total product cost and the number
of sold copies is estimated at 500.

\begin{equation}
\begin{split}
GP &= C_{total} / N_{cs} + P_{p}\\
    &= 116669/500 \cdot 1.15 \\
    &= 268.34,
\end{split}
\end{equation}

\noindent where $GP$ is the gross price, $C_{total}$ -- total product cost,
$N_{cs}$ -- number of copies sold, $P_{p}$ -- chosen profit percentage. This
is not the price of the end product, since it is necessary to add sales tax
(VAT), which represents $20\%$ and is added to the gross price.

\begin{equation}
\begin{split}
P_{sale} &= GP + TX_{sales}\\
            &= 268.34 \cdot 1.20 \\
            &= 322.00,
\end{split}
\end{equation}

\noindent where $P_{sale}$ is the sale prices including VAT, $GP$ -- gross
price, $TX_{sales}$ -- sales tax. The net income is computed by multiplying
gross price and the number of expected copies to be sold, which will be

\begin{equation}
\begin{split}
I_{net} &= GP \cdot N_{cs}\\
        &= 268.34 \cdot 500 \\
        &= 134170,
\end{split}
\end{equation}

\noindent where $I_{net}$ is the net income, $GP$ -- gross price, $N_{cs}$ --
number of copies sold. Moreover it is necessary to compute the gross and net
profit. The indicators are $GPr$ -- gross profit and $NPr$ -- net profit.

\begin{equation}
 \begin{split}
GPr &= I_{net} - C_{production}\\
    &= 134170 - 116669\\
    &= 17501 \\
NPr &= GPr - 12\% \\
    &= 17501 \cdot 0.88 \\
    &= 15400.88,
\end{split}
\end{equation}

\noindent where $I_{net}$ is the net income, $C_{production}$ -- cost of
production. The profitability indicators are $C_{profit}$ -- cost
profitability, $S_{profit}$ -- sales profitability computed as percentage.

\begin{equation}
 \begin{split}
  C_{profit} &= GPr / C_{production} \cdot 100\%\\
              &= 17501 / 116669 \cdot 100\% \\
              &= 15.0 \%\\
  S_{profit} &= GPr / I_{net} \cdot 100\% \\
             &= 17501 / 134170 \cdot 100\% \\
             &= 13.0 \%.
 \end{split}
\end{equation}

\subsection{Marketing Plan}

After the product has come to the end of the production process, it needs to
be sold to clients. This, however, is not as simple a task as it sounds. In
order to find the right audience and choose the right way to offer the
product, a series of activities should be performed. This set of activities is
called \emph{marketing} and is here to guide the flow of products from the
producer to the clients. Marketing is a system of economical activities about
price setting, promotion and distribution of products and services to satisfy
current and potential consumers requests. Marketing is the science and art of
exploring, creating, and delivering value to satisfy the needs of a target
market at a profit.

Functions of Marketing:
\begin{itemize}
    \item Analysis of external environment;
    \item Analysis of consumer behavior;
    \item Development of product;
    \item Development of distribution;
    \item Development of promotion;
    \item Price setting;
    \item Social responsibility;
    \item Management marketing.
\end{itemize}

The game that is at the focal point of this thesis might be a hard thing to
commercialize, but there are some ways in which it could bring profit. In
order to identify the direction in which a product development path should
lay, it is necessary to perform an extensive market research. Even though
market research should be present to some extent in the whole development
cycle, its initial stage is expected to provide a starting reference point for
the product.

Market research stages:
\begin{itemize}
    \item Identifying the problem;
    \item Developing program of research and gathering;
    \item Establishing specific information ( internal, external );
    \item Establishing methods for collecting data;
    \item Performance of research;
    \item Information analysis, drawing conclusions.
\end{itemize}

The very beginning of the product evolution could be the most expensive for a
company launching a new product. The cost of development is quite high in this
period while the market is still very small. In case the product finds success
on the market, it usually experiences a stage of strong growth in sales and
profits, this in turn triggers the activity of the company as now it can cover
easily the development expenses as well as invest in promotional activity.
This results in profit maximization and the realization of the full potential
of the product. When the product matures, the main goal of the company is to
maintain the product on the market as at this moment a lot of competitors are
coming out of shadows. In such conditions product managers must come up with
improvements and modifications to the product, that would give it a
competitive advantage. After this, usually, comes the declining stage in which
the market of the product shrinks as a result of market saturation and sales
start to become smaller as well. At this point the product is most likely
evolve into a completely different one.


\subsection{Economic Conclusions}

The commercialization of a product is definitely not an easy task and the
'Snowfight' game is no exception to this observation. Moreover, the game
itself is intended as an illustration of another product, the communication
framework and holds a rather small value by itself. On the other hand it can
serve as a gateway into a market of such kind of applications.

This section has exposed an extensive economic analysis of the game as a
standalone product. It gave a perspective on the amounts of potential expenses
that might be involved in the production process of the game as well as an
estimate of how well the product should be sold in order to cover its costs
and, on top of that, bring profit.

As a result of this analysis it became clear that in order to perform such a
project in a commercial environment it is crucial that an exhaustive market
research is performed in order to find the right niche for such a product and
protect the company from failure. Still, at this stage it can be observed that
'Snowfight' might be suitable for the entertainment business and if it cannot
bring money, it sure can bring fun.

\clearpage
