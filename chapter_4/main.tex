\section{Economic Analysis}
\phantomsection

\subsection{Project Description}

The focus point of this thesis is the development of a toolkit that would
enable software engineers to create with ease web-based applications that
feature real-time communication with mobile devices and leverage all the
sensors and feedback elements that a modern smart-phone can deliver. In the
economic context, however, it is not enough to develop a framework, it is also
crucial to have a demonstration application that implements logic which uses
this framework, so that it can gain public attention and obtain necessary
financial investments to support its further development. The 'Snowfight' game
is a wonderful way to illustrate the power of real-time peer-to-peer
interaction between a desktop browser and a browser of a mobile device. It can
also effectively showcase the majority of the use-cases of the framework.

\subsection{Project Time Schedule}

In order to be successful, a project needs to have an effective development
methodology. Presently, \emph{Agile Software Development} gains popularity and
proves to be a flexible and robust way to create software. This methodology
implies an iterative and adaptive process of development and this in turn
influences the time table that has to be established.

\subsubsection{Objective Determination}

The main objective of the project can be derived from its motivation. The
'Snowfight' should illustrate at its best the features of the toolkit
described in the thesis. The game should represent an isometric multiplayer
arcade in which participants can throw snowballs in each other and score
points. The players should control their characters via their smart-phones,
specifically through a web page rendered in the browser of the phone which
represents a some kind of game controller. The gameplay should feel very
responsive as if the player uses a wired gamepad.

Besides showing off technical features of the framework, the game should
appeal to potential players that would use it as an entertainment asset at
parties or team-building events. It should be engaging and well balanced as to
provide casual, yet challenging, and maybe even addictive, gameplay. The
controller part of the system be supported by most types of mobile devices.

\newpage
\subsubsection{SWOT Analysis}

In order to get a better picture of the product and different paths of its
development it is a great idea to analyze the project from the perspective of
Strengths, Weaknesses, Opportunities and Threats, as it has proven to be an
effective technique in project management.

\paragraph{Strengths}

\begin{itemize}
    \item Intuitive user interface
    \item Multiplayer with up to 10 players
    \item Doesn't require any special programs to be installed
    \item Entertaining
\end{itemize}

\paragraph{Weaknesses}

\begin{itemize}
    \item The technology is available only on mid to high-end devices
    \item The game is somewhat bound by the limitations of the graphical framework
    \item Only one game mode
\end{itemize}

\paragraph{Opportunities}

\begin{itemize}
    \item Add several game modes
    \item Improve graphics art
    \item Use this project as a step-stone for another project
\end{itemize}

\paragraph{Threats}

\begin{itemize}
    \item Lack of visibility on the Internet
    \item Development of similar frameworks by other teams
\end{itemize}

\subsubsection{Time Schedule Establishment}

Most of IT projects' lifetimes consist of 5 basic steps: planning, research,
development, testing and deployment. These steps are in turn subdivided into
smaller parts in order to make the whole process manageable in terms of tasks.
Most of the time table will be allocated to the step of development and
testing as these are the steps with the most workload. Planning and research
by itself shouldn't take up too much time, as requirements usually evolve
during development, also research is a thing that usually never stops. The
development of the project can also be split in tasks that can be performed in
parallel thus minimizing the overall time. Total duration of the project is
computed using the equation \eqref{eq:duration}.

\begin{equation} \label{eq:duration}
D_T = D_F - D_S + T_R,
\end{equation}

\noindent where $D_T$ is the duration, $D_F$ -- the finish date, $D_S$ -- the
start date and $T_R$ -- reserve time. Table \ref{table:schedule} represents
the first iteration of the project schedule. The following notations are used
to improve readability: PM -- project manager, D -- developer, GD -- graphics
designer, SM -- sales manager.

\begin{table}[!h]
\begin{center}
\caption{Time schedule}
\renewcommand{\arraystretch}{1.5}
\begin{tabulary}{1\textwidth}{| C | C | C | C | C |}

\hline \textbf{Nr} & \textbf{Activity Name} & \textbf{Duration (days)} & \textbf{People involved} \\
\hline 1  & Project concept definition          & 5     & PM, GD, SM, D \\
\hline 2  & Market analysis                     & 10    & PM, SM        \\
\hline 3  & Domain analysis                     & 15    & PM, D         \\
\hline 4  & Product requirements specification  & 5     & PM, D         \\
\hline 5  & System modeling (UML)               & 10    & D             \\
\hline 6  & Graphic design                      & 10    & GD            \\
\hline 7  & Framework development               & 20    & D             \\
\hline 8  & Main game development               & 30    & PM, GD, D     \\
\hline 9  & Validation of results               & 10    & PM, GD, D, SM \\
\hline 10 & Documentation                       & 5     & D             \\
\hline 11 & Deployment and testing              & 10    & PM, D         \\
\hline 12 & Active marketing                    & 15    & SM            \\
\hline    & Total time to finish the system     & 145   &               \\
\hline

\end{tabulary}
\label{table:schedule}
\end{center}
\end{table}

\newpage
\subsection{Economic Motivation}

This section aims to give a perspective of the project from the economic point
of view. This includes the expenses and profits encountered as a result of the
project's activity as well as the various strategies of commercializing. All
the costs and prices are given in MDL (Moldavian lei) currency. Tangible and
intangible assets, indirect expenses will also be taken into account. One
important thing to mention is that the game is the fact that it is being
developed as an open-source project and it's primary goal is to popularize the
framework and toolkit underneath. Nevertheless, there are still opportunities
to commercialize the game, especially on the rising wave of the term
'Gamification' which gained substantial popularity lately in various
industries. Thus, the game in modified versions can be licensed to companies
that would like to 'gamify' their internal processes. Another monetization
model that might bring some profit and cover the development expenses is the
so called 'freemium' pricing model. This implies that the game should be
released for free with a limited set of features and the full package
of features would be unlocked to people that purchase a subscription of some
kind.


\newpage
\subsubsection{Asset Expense Evaluation}

In order to compute the potential cost of the product it is necessary to
evaluate the cost of various assets involved in the production of the project.
There are tangible and intangible assets that impose certain costs for the
project, the budget for both types of assets are exposed in tables
\ref{table:tangible_assets} and \ref{table:intangible_assets}.

\begin{table}[!h]
\begin{center}
\caption{Tangible asset expenses}
\renewcommand{\arraystretch}{1.5}
\begin{tabulary}{1\textwidth}{| C | C | C | C | C |}

\hline \textbf{Material} & \textbf{Specification}  & \textbf{Price per unit (MDL)} & \textbf{Quantity} & \textbf{Sum (MDL)}\\
\hline Workstation & Apple Macbook Pro 13' Retina   & 33000  & 2     & \multicolumn{1}{r|}{66000} \\
\hline Smartphone  & Nexus 5 and iPhone 5s          & 9000   & 2     & \multicolumn{1}{r|}{18000}  \\
\hline WebHosting  & Microsoft Azure VM             & 10000  & 1     & \multicolumn{1}{r|}{10000}  \\
\hline \multicolumn{4}{|r|}{Total}                                   & \multicolumn{1}{r|}{96000} \\
\hline
\end{tabulary}
\label{table:tangible_assets}
\end{center}
\vspace{-1.3em}
\end{table}

\begin{table}[!h]
\begin{center}
\caption{Intangible asset expenses}
\renewcommand{\arraystretch}{1.5}
\begin{tabulary}{1\textwidth}{| C | C | C | C | C |}

\hline
\textbf{Material} & \textbf{Specification} & \textbf{Price per unit (MDL)} & \textbf{Quantity} & \textbf{Sum (MDL)} \\
\hline License      & Enterprise Architect Desktop Edition License   & 1900  &   1   & \multicolumn{1}{r|}{1900} \\
\hline License      & Sublime Text 3 License                         & 1500  &   1   & \multicolumn{1}{r|}{1500} \\
\hline License      & Adobe Photoshop CC 2016                        & 2000  &   1   & \multicolumn{1}{r|}{2000} \\
\hline Domain Name  & Domain name registration fee                   & 500   &   1   & \multicolumn{1}{r|}{500} \\
\hline \multicolumn{4}{|r|}{Total}                                                   & \multicolumn{1}{r|}{4900} \\
\hline
\end{tabulary}
\label{table:intangible_assets}
\vspace{-1em}
\end{center}
\end{table}


\newpage
\subsubsection{Direct Expenses}

Besides the costs of assets, the development of the project implies some
direct expenses like consumable materials for the office. Even though the
amounts are  not that big, is important to keep track such expenses as they
tend to add up. These direct expenses are presented in the table
\ref{table:direct_expenses}.

\begin{table}[!h]
\begin{center}
\caption{Direct expenses}
\renewcommand{\arraystretch}{1.5}
\begin{tabulary}{1\textwidth}{| C | C | C | C | C |}

\hline \textbf{Material} & \textbf{Specification} & \textbf{Price per unit (MDL)} & \textbf{Quantity} & \textbf{Sum (MDL)} \\
\hline Whiteboard      & Universal Dry Erase Board & 500    & 1     & \multicolumn{1}{r|}{500} \\
\hline Paper           & A4 - 500 sheet pack       & 60     & 2     & \multicolumn{1}{r|}{120} \\
\hline Marker          & Whiteboard marker         & 15     & 10    & \multicolumn{1}{r|}{150} \\
\hline Pen             & Blue pen                  & 5      & 10    & \multicolumn{1}{r|}{50} \\
\hline \multicolumn{4}{|r|}{Total}                                  & \multicolumn{1}{r|}{820} \\
\hline
\end{tabulary}
\label{table:direct_expenses}
\vspace{-1.5em}
\end{center}
\end{table}


Given the tables above the total amount of expenses in MDL, including tangible
and intangible asset expenses, is the following:

\begin{equation}
T_{e} = 96000 + 4900 + 820 = 11720
\end{equation}


\subsubsection{Salary Expenses}

It is obvious that a major part of project expenses go for the salaries of the
employees. This section describes the computations concerning salary funds.
The starting point is the distribution of per day salaries that the employees
are expected to receive. It looks like this: project manager - 400MDL, sales
manager - 300 MDL, developer - 380 MDL, graphics designer - 350 MDL.

\begin{table}[!ht]
\begin{center}
\caption{Salary expenses}
\renewcommand{\arraystretch}{2}
\begin{tabulary}{1\textwidth}{| C | C | C | C |}

\hline
\textbf{Employee} & \textbf{Work fund (days)} & \textbf{Salary per day (MDL)} & \textbf{Salary fund (MDL)} \\
\hline Project Manager      & 80    & 400   & \multicolumn{1}{r|}{32000} \\
\hline Developer            & 110   & 380   & \multicolumn{1}{r|}{41800} \\
\hline Graphics Designer    & 55    & 350   & \multicolumn{1}{r|}{19250} \\
\hline Sales Manager        & 40    & 300   & \multicolumn{1}{r|}{12000} \\
\hline
\multicolumn{3}{|r|}{Total}                 & \multicolumn{1}{r|}{105050} \\
\hline
\end{tabulary}
\label{table:salaries}
\end{center}
\end{table}

Given the data in the table \ref{table:salaries} it is necessary to compute the amounts of
money that should be payed to social services fund and medical insurance fund.
After this it is possible to sum everything up and obtain the total work
expense amount.

This year the social service fund is approved to be $23\%$, therefore the
salary expenses are computed according to the relation \eqref{eq:fs}.

\begin{equation}\label{eq:fs}
\begin{split}
 FS &= F_{re} \cdot T_{fs} \\
    &= 105050 \cdot 23 \% \\
    &= 24161.50,
\end{split}
\end{equation}

\noindent where $FS$ is the salary expense, $F_{re}$ is the salary expense
fund and $T_{fs}$ is the social service tax approved each year. The medical
insurance fund is computed as

\begin{equation}
\begin{split}
 MI &= F_{re} \cdot T_{mi}\\
    &= 105050 \cdot 4.5\%\\
    &= 4727.25,
 \end{split}
\end{equation}

\noindent where $T_{mi}$ is the mandatory medical insurance tax approved each
year by law of medical insurance and this year it is $4.5\%$.

The total work expense fund can now be computed, given the amounts of social
service and medical insurance taxes.

\begin{equation}
\begin{split}
 WEF &= F_{re} + FS + MI\\
     &= 105050 + 24161.50 + 4727.25\\
     &= 133928.75,
\end{split}
\end{equation}

\noindent where $WEF$ is the work expense fund, $FS$ is the social fund and $MI$
is the medical insurance fund. In that way the total work expense fund was
computed.


\subsubsection{Individual Person Salary}

Along with total work expense fund, it is necessary to compute the annual
salary for the developer. Considering that the developer has a salary of 380
MDL per day and there are totally 250 working days in a year, so the gross
salary that the developer gets is:

\begin{equation}
GS = 380 \cdot 250 = 95000,
\end{equation}

\noindent where $GS$ is the gross salary computed in MDL.

Social fund tax this year represents $6\%$, so the amount that should be tax
paid in MDL represents:

\begin{equation}
 SF = 95000 \cdot 6\% = 5700.
\end{equation}

Medical insurance tax represents $4.5\%$ and gives the following result:

\begin{equation}
 MIF = 95000 \cdot 4.5\% = 4725.
\end{equation}

In order to proceed with income tax computations, it is necessary to calculate
the amount of taxed salary:

\begin{equation}
\begin{split}
 TS &= GS - SF - MIF - PE \\
              &= 95000 - 5700 - 4725 - 10128\\
              &= 74447,
\end{split}
\end{equation}

\noindent where $TS$ is the taxed salary, $GS$ -- gross salary, $SF$ -- social
fund, $PE$ -- personal exemption, which this year is approved to be $10128$.

The last but not the least thing to be computed is the total income tax, which
is $7\%$ for income under 29640 MDL and $18\%$ for income over 29640 MDL:

\begin{equation}
\begin{split}
 IT &= TS - ST \\
      &= 29640 \cdot 7\% + (74447 - 29640) \cdot 18\% \\
      & = 2074.80 + 8065.30 = 10140.10,
 \end{split}
\end{equation}

\noindent where $IT$ is the income tax, $TS$ -- the taxed salary and $ST$ --
the salary tax. With all this now it is possible to find out the value of the
net income:

\begin{equation}
\begin{split}
 NS &= GS - IT - SF - MIF \\
            &= 95000 - 10140.10 - 5700 - 4725 \\
            &= 74434.90,
\end{split}
\end{equation}

\noindent where $NS$ is the net salary allocated for the employee, $GS$ -- gross salary, $IT$ -- income
tax, $SF$ -- social fund, $MIF$ -- medical insurance fund.


\subsubsection{Indirect Expenses}

\subsubsection{Wear and Deprecation}

\subsubsection{Product Cost}

\subsubsection{Economic Indicators and Results}

\subsection{Marketing Plan}

\subsection{First Clients}

\subsection{Economic Conclusions}






\clearpage
