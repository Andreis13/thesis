
\documentclass[12pt,a4paper,titlepage]{article}
\usepackage{License_style}


\usepackage[style=numeric-comp, backend=bibtex, sorting=none]{biblatex}
\addbibresource{thesis.bib}

\begin{document}

\numberwithin{lstlisting}{section}

\setcounter{page}{4}

\phantomsection
%\addcontentsline{toc}{section}{SUMMARY}
\input{misc/summary.tex}
\cleardoublepage

\phantomsection
%\addcontentsline{toc}{section}{АННОТАЦИЯ}
\section*{Rezumat}

\selectlanguage{romanian}

Această teză reprezintă o încercare de a utiliza tehnologiile web moderne și
anume acelea din domeniul comunicării în timp real, cu scopul de a consolida și
de a încuraja comunicarea și interacțiunea între oameni. Scopul va fi realizat
prin dezvoltarea unui joc multiplayer bazat pe web ce utilizează telefoanele
mobile în calitate de dispozitiv de control, emulînd experiența unei
console de jocuri.

Textul tezei este format din patru capitole, care cuprind etapele individuale a
procesului de dezvoltare a proiectului. Capitolul 1 efectuează o analiză
detaliată a domeniului, oferă un scurt istoric a modului în care a evoluat
interacțiunea omului cu calculatorul și identifică motivația pentru proiectul
tezei. În același capitol sunt prezentate potențialele tehnologii (WebSockets și
WebRTC) ce pot fi utilizate pentru rezolvarea problemei în cauză. Ultimele
secțiuni ale primului capitol conțin o analiză a aplicațiilor care sunt în
prezent pe piață.

Capitolul 2 este axat pe aspectele arhitecturale ale sistemului. El include
rezultatele procesului de modelare în formă de diagrame UML, însoțite de
descrierea sistemului și a părților individuale a acestuia din diferite
perspective. La fel sunt specificate deciziile de proiectare, împreună cu
motivațiile care au stat în spatele lor.

Al treilea capitol oferă o perspectivă detaliată asupra implementării
proiectului. Trei subsecțiuni majore descriu părțile respective ale sistemului,
și anume: jocul, componenta dispozitivului de joc și codul care realizează
comunicarea între acestea. La fel sunt incluse fragmente de cod cu comentarii ce
accentuează tehnicile și instrumentele care au fost folosite în timpul
dezvoltării.

Ultimul capitol realizează o trecere în revistă a proiectului din punct de
vedere economic cu încercarea de a estima potențialul său economic. Acesta
include calcule de diferite tipuri de cheltuieli care în ansamblu constituie
costul total al produsului. Acesta este folosit pentru a calcula indicatorii
economici ai proiectului și prețul de vânzare a unei unități de
produs în cazul în care acesta ar fi introdus pe piață.

La finalul tezei sunt prezentate rezultatele generale a proiectului care au fost
obținute în timpul lucrului cu tehnologiile utilizate în proces de dezvoltare.

\selectlanguage{english}

\clearpage

\cleardoublepage


\tableofcontents
\addtocontents{toc}{\protect\thispagestyle{empty}} % no page number on the table of contents page
\cleardoublepage

% LISTA FIGURILOR. Este recomandabila daca in text ai peste cel putin 10-15 FIGURI
\listoffigures
\addcontentsline{toc}{section}{List of figures}
\clearpage

% LISTA TABELELOR. Este recomandabila daca in text ai peste cel putin 10-15 tabele
\listoftables
\addcontentsline{toc}{section}{List of tables}
\clearpage

\lstlistoflistings
\addcontentsline{toc}{section}{Listings}
\clearpage

% ABREVIERI. Este recomandabila doar daca utilizezi in text peste 10-15 abrevieri
% \phantomsection
% \addcontentsline{toc}{section}{Abbreviations}
% \input{misc/abbreviations.tex}
% \cleardoublepage


% Introduction
%\setcounter{page}{17}
\phantomsection
\addcontentsline{toc}{section}{Introduction}
\phantomsection
\addcontentsline{toc}{section}{Introduction}
\section*{Introduction}




\clearpage

\cleardoublepage

% Chapter 1
\section{Domain Analysis}

\subsection{User Interfaces and Input Devices}

% Before diving in the deeps of real-time communication over the Web, it is necessary to review the path that people took in order to interact with computers in the first place.

At the very beginning of the computer era, the devices that we now know as
PCs, laptops and tablets were immense pieces of machinery that occupied entire
laboratories and only highly qualified individuals had access to them. These
machines where developed in order to perform various numerical problems by
executing digital computations. As the saying goes, a machine can only do what
a man tells it to do, so engineers had to come up with different means to set
tasks for these electronic beasts, that is, a need for \emph{user interfaces}
arouse. Operators used large stacks of punched cards to feed instructions and
data sets to computers. These punched cards in turn where created using
specialized devices that also required some knowledge in the field. Over all
it was a complex process that couldn't be easily grasped by an ordinary
persons. At that time, however, the majority of computer user had PhDs and
were trained to perform these very tasks so the difficulty of communicating
with machines wasn't really a great problem.

% picture of ENIAC maybe

In time, the interest towards computers grew bigger among enthusiasts while
the devices themselves where becoming smaller and smaller, eventually giving
birth to the term of 'microcomputer'. Among the first microcomputers to get
widespread popularity was the critically acclaimed Altair 8080 which
represented a box with lights and switches on the front panel that where used
to feed data and instructions into the computer and read the results back from
it.

% picture of Altair 8080 front panel

As more hobbyists took a hold of such devices people discovered many uses for
them beyond that of performing various mathematical operations. They learned
how to connect teletypes to computers and this way the later got an interface
that is familiar to all of us today, a keyboard. For some time people
interacted with computers using command lines, but with the development of
computer graphics and the creation of graphical user interfaces, a need for a
new kind of controller appeared, specifically the need for a pointing device.
Since then, a standard personal computer included a mouse and a keyboard among
it's peripheral devices.

\subsubsection{Input Devices for Gaming}

On the other hand, the development of input devices never stopped with the
creation of the mouse and keyboard. A wealth of specialized devices was
created to perform tasks that were specific to a particular use of the
computer. One of the most notable fields that was moving the development of
specialized peripherals was and still is the \emph{game development industry}.
During the years, various types of controllers were developed, some of them
are listed below:

\begin{description}

	\item [Gamepad] is a general-purpose gaming device that is used as a
		controller for a wide range of game genres, from arcades and fighting
		games to role-playing titles and action third-person shooters.

	\item [Joystick] is a specialized controller that is often used in flight
		simulation games, although its use can be extended far beyond gaming, for such
		purposes as remote control of a robot arm in a warehouse.

	\item [Steering Wheel] is used to provide almost authentic driving
		experience in racing video games and simulators. \end{description}

% picture of stuff mentioned above

% include info about Nintendo Wii

\subsection{Mobile User Interfaces}

Very soon, technological progress made it possible to stick microcomputers
virtually in any device, be it a fridge, microwave oven or a car. The most
benefit out of this, however, gained the industry of mobile phones, eventually
giving birth to the concept of smart-phones. Today, almost every person living
in a more or less civilized region of the world has in his/her pocket a little
'brick' with the computer power thousands of times greater than the one that
was used to send people to the moon. This breakthrough in portable computing
facilitated the development of software for such devices and the majority of
companies started to support mobile applications where it made sense.

In the context of user interface design and input handling techniques,
developers found themselves in a completely different world. Nobody would
carry a mouse and a keyboard in order to interact with the phone, instead,
people would use the number pad plus a couple of additional buttons present on
the phone. In time, another revolution took place, it was the moment when
touch screens became reality. Once again, developers had to think their way
through these changes and create user interfaces that would suit fingers
poking into the screen rather than button clicks. With the difficulties of
user interface redesign for touch devices came also a wide range of new
possibilities. Now that the whole screen constituted an input device, the
users could perform gestures like taps, swipes, pan motions, pinch-zoom
actions. These gestures made the process of using an application a great deal
more intuitive when compared to the times when every action was performed
through the click of a certain button.

Besides touch screens, mobile phone vendors began to include in their devices
such sensors like accelerometers and gyroscopes. Accelerometers are used to
catch the moment when the device is moved and to calculate the direction and
acceleration of such movement. On the software side this could be used in
various ways, the simplest being to switch to the next song in the music
player by jerking the phone to the right. Meanwhile, gyroscopes where designed
to capture the orientation of the device at a given moment in time. This
functionality is employed every time the phone is tilted over 90 degrees and
triggers the switch between landscape and portrait view modes.

When it came to the industry of mobile gaming, touch screens and sensors that
could capture the motion and orientation of the device were a really big deal,
because they opened the possibilities for developing complex user interfaces
that would simulate specialized gaming devices. The accelerometer and
gyroscope would be used as the steering wheel in racing games while the touch
screen would capture taps in various regions and interpret them as clicks of
different buttons of a gamepad, whose simplified image would be rendered on
the same screen.

% picture of mobile game UI which resembles a gamepad

\subsection{Mobile Devices as Controllers for Desktop Computers}

When comparing the features of a specialized gaming device like \emph{Nintendo
Wii} Controller and the capabilities of a modern smart-phone, an interesting
pattern of similarities can be observed. Both are able to capture device
motion and position. Both respond to clicks of buttons in case of a gamepad
and taps in designated areas in case of a phone. Touch screens can be used to
simulate even the motion of a joystick, by capturing swipe or pan gestures.
One feature that a specialized gaming device may have, that a phone falls
short of, is ergonomics, but to a certain extent this can be neglected. With
such a list of similarities, a logical question appears: Why not use mobile
phones as gaming controllers for desktop games?

\subsubsection{The Problem}

As it turns out, now that smart-phones run full-fledged operating systems,
connecting one to a computer is not that hard of a task. Mobile operating
systems like Android, Apple iOS and Windows Mobile, all support socket
programming, thus a programmer can setup a communication channel over the
network between a mobile phone and a laptop connected to the same Wi-Fi
hotspot. They can even be located on different sides of the planet, it is
sufficient to have an Internet connection to be able to exchange data between
devices.

Overall it looks like a way to go approach. On the other hand, there are some
problems with it that are not to be observed at first sight. In the first
place, every time that a company wants to create a game with support for
mobile devices as controllers, the developers will have to write custom mobile
applications in addition to the main game, they will also have to devise
specialized communication protocols. This is both, time and money consuming
and is not economically convenient. Secondly, there is a problem on the user
side which is best illustrated with a situation. Let's suppose a party or a
team-building event where the host tries to entertain his guests with a
multiplayer computer game. Not everyone has at home a gaming platform like
Sony Playstation or Microsoft Xbox, neither does anyone have more than 2-3 PC
gamepads. In this situation, the ability to use smart-phones as controllers
would be a great benefit. In case the connection is implemented in the way
described above, a gaming party would transform into a setup party, where
guests would spend a lot of time installing mobile applications that are
probably needed only for one-time use and don't hold any value by themselves
without the main game. Fortunately both problems can be solved relatively
simple through the same solution.

\subsubsection{The Solution}

One thing that all modern smart-phones have in common is a decent web browser.
Today, browsers are more than just viewers of HTML documents, they are entire
ecosystems and programming environments that are almost independent of the
underlying operating system. The fact that every mobile phone has such an
environment makes it possible to write cross-platform application served on
the web that would otherwise be installed manually as a native app. Modern
browsers support APIs that can interact with various parts of a mobile device
like accelerometer, gyroscope and touch screen, exactly the things that are
necessary to simulate a fully functional gaming device.

Having said this, one thing that could solve the problems stated in the
section above may be a framework or toolkit that unifies in a single library
all the APIs that are necessary to connect a mobile phone to a computer as a
remote gaming controller. It would make it easier and faster to develop games
with such a feature. This toolkit may also include user interface building
blocks which can be combined to create custom controllers.

The purpose of this thesis is to create a simple web-based multiplayer
isometric arcade called 'Snowfight'. The main game is started by accessing its
web page. The first thing the players see is a lobby and a connection URL.
Players join the game by navigating to the provided URL using the browsers
installed on their phones. When enough players have connected and the
participants decide to start the game, a button can be clicked on the main
game screen. The game play concept is rather simple, players can move their
characters around the game space and throw snowballs in each other using the
controls rendered by the web application that works on their phones. Every
player has a certain amount of hit points (HP) and if a player is hit, his HP
amount decreases. When a player's HP amount reaches zero, he is eliminated
from the game. The goal of the game is to eliminate all players from the
adversary team.

In order to implement this project it is necessary to explore the topic of
real-time communication in web, as to be able to chose the right technology
that would provide a smooth and responsive gaming experience.




\subsection{Real-Time Communication over the Web} % about communication technologies

One of the main uses of the Internet is hosting and accessing web pages. When a
user writes a URL in the browser's search bar and presses 'enter', the browser
preforms an HTTP request to the server specified by the URL and fetches a web
page. At this point in time, for most cases, the communication between the
server and the client (browser) is ceased until the user clicks on another link
that would restart the process.

On the other hand, modern web has greatly developed during the past few years.
Internet connection speed has increased dramatically, and many websites have
stopped being just static web pages and have moved towards a more dynamic model.
They are not even called websites, today, all around the world there are web
applications.

With the requirements for a more dynamic web came also the technical challenges
of making it so. For instance, how would a user's browser receive a notification
about a message that was sent to this user from another part of the globe? When
such questions just appeared, developers tried different techniques using the
old tools they had at hand at that time. The concepts of HTTP Long Polling and
HTTP Streaming appeared in this period. Soon, however, they were found to be
causing quite a bit of issues\cite{long_polling_issues} and a need for custom
protocols arose.

As the web is mainly composed of clients and servers, the logical outcome was a
protocol that would connect them and would permit bidirectional communication
between the server and its clients, thus WebSockets were proposed. At the same
time, browsers grew and their uses extended. Soon, real-time communication out-
sized the context of client-server architecture and when video chat applications
and multiplayer games started to conquer the web realm, engineers developed
WebRTC, which solved the issue of communication in peer-to-peer setups. A brief
description of both protocols as per their RFCs is presented in the sections
that follow.

\subsubsection{WebSockets for Client-Server Communication} % Websockets

Historically, creating web applications that need bidirectional communication
between a client and a server (e.g., instant messaging and gaming applications)
has required an abuse of HTTP to poll the server for updates while sending
upstream notifications as distinct HTTP calls.

As RFC6455\cite{websockets} points out, this results in a variety of problems:

\begin{itemize}

\item The server is forced to use a number of different underlying TCP
  connections for each client: one for sending information to the
  client and a new one for each incoming message.

\item The wire protocol has a high overhead, with each client-to-server
  message having an HTTP header.

\item The client-side script is forced to maintain a mapping from the
  outgoing connections to the incoming connection to track replies.

\end{itemize}

A simpler solution would be to use a single TCP connection for traffic in both
directions. This is what the WebSocket Protocol provides. Combined with the
WebSocket API, it provides an alternative to HTTP polling for two-way
communication from a web page to a remote server.

The same technique can be used for a variety of web applications: games, stock
tickers, multiuser applications with simultaneous editing, user interfaces
exposing server-side services in real time, etc.

The WebSocket Protocol is designed to supersede existing bidirectional
communication technologies that use HTTP as a transport layer to benefit from
existing infrastructure (proxies, filtering, authentication). Such technologies
were implemented as trade-offs between efficiency and reliability because HTTP
was not initially meant to be used for bidirectional communication. The
WebSocket Protocol attempts to address the goals of existing bidirectional HTTP
technologies in the context of the existing HTTP infrastructure; as such, it is
designed to work over HTTP ports 80 and 443 as well as to support HTTP proxies
and intermediaries, even if this implies some complexity specific to the current
environment. However, the design does not limit WebSocket to HTTP, and future
implementations could use a simpler handshake over a dedicated port without
reinventing the entire protocol. This last point is important because the
traffic patterns of interactive messaging do not closely match standard HTTP
traffic and can induce unusual loads on some components.


\subsubsection{WebRTC for Peer to Peer Communication} % WebRTC

\newpage

% \subsection{Mobile devices Game Controllers} % game controllers + web

% \subsubsection{Native Apps}

% \subsubsection{Web Apps}

\subsection{Solutions on the Market}

\subsubsection{Lightsaber Escape} % https://lightsaber.withgoogle.com/

\newpage

\subsubsection{Super Sync Sports} % https://www.chrome.com/supersyncsports/

\newpage

\subsection{Project Description}

\newpage

\subsection{Domain Analysis Conclusions}

\newpage

\clearpage

\cleardoublepage

% Chapter 2
\section{System Modeling}
\phantomsection

\subsection{Use Case Layout}

\subsection{User Activity Graph}

\subsection{Player States}

\subsection{High-level Game States}

\subsection{Class Hierarchy of Game Objects}

\subsection{Components and Libraries}

\subsection{Deployment Setup}

\subsection{System Design Conclusions}

\clearpage

\cleardoublepage
% \addtocontents{toc}{\protect\newpage}

% Chapter 3
\section{System Implementation Details}
\phantomsection

In this chapter is exposed a detailed description of the implementation details
of the thesis project. In the following sections are presented script excerpts
from both, the game and the controller parts of the system as well as the
communication code that links them together.

\subsection{Main Game}

The development of computer games is one of the most complex branches of the
software industry. Games encompass about a dozen different fields of mathematics
and computer sciences, to name a few, there is three-dimensional calculus,
differential calculus for physics, computer graphics, artificial intelligence,
game theory, etc.

Due to its complexity, game programming is among fields where the concept of
libraries and frameworks is more relevant than ever. Along the years, developers
around the globe have stumbled on the same problems in a recurring manner and as
a result, a lot of experience was gained which today is materialized as game
engines and development kits for all kind of platforms and types of games. In
this context it is much easier to prototype a game than it was ten years ago.

This thesis uses an HTML5 game framework called \emph{Phaser} which features a
rich set of tools that are required to build a 2D video game. This includes
rendering to an HTML Canvas or WebGL context, a physics engine, particle system,
asset management framework, sound engine, input handling, tiles, sprites and
much more. This framework was selected due to the ease and speed with which one
can develop a working prototype game that looks nice enough to be presentable.
On the other hand, 'Snowfight' is designed as an isometric game and doesn't meet
the main purpose of the framework, but this problem is easily solved by making
use of the framework's powerful plug-in system. The isometric plug-in extends
the Phaser capabilities like physics into the three-dimensional world and
perform an isometric projection on a two-dimensional canvas afterwards.

% TODO link to phaser isometric


\subsubsection{The Design of a PhaserJS Game}

In order to use a tool efficiently it is important to understand how it works.
The Phaser framework follows classic game programming patterns and has a
standard game loop which consists of three main steps:

\begin{description}

\item [Input handling] - the part where the user input is collected and
transformed into actions that are applied to the world whether it is changing
the direction and velocity of the player's movement or throwing a snowball

\item [Simulation update] - represents the process of updating the parts of the
system that do not directly depend on user input, like collision resolution,
object position adjustment depending on its velocity and such things as choosing
and applying the right frame of a sprite animation

\item [Rendering to screen] - consists of drawing all elements and textures on
an HTML5 Canvas or a WebGL context. When using Phaser, the programmer would
seldom override this step as the framework already does most of the work, with
of some rare cases when very specific adjustments and post-processing is needed

\end{description}

In Phaser games, the loop itself is hidden under the hood of the framework,
while a simple yet flexible interface is provided to the programmer to control
and fine-tune the steps specified above. In order to bootstrap a game it is
enough to instantiate a Phaser::Game object while providing the necessary
callbacks as in the example below:

\lstinputlisting[caption=Minimal Setup for a Phaser Game]{phaser_minimal.js}

The three methods in the example are not the only ones that can be overridden
and Phaser documentation list all of them and explains how they can be used to
customize the various use cases. The next section not only presents these
methods, it describes how different versions of them can be specified to be used
in different context, thanks to the concept of game states.


\subsubsection{Game State Management}

Every game is like a living system and everything a user sees happens inside a
game loop. Even if the screen shows the same static image, the game loop still
runs and refreshes the screen about 60 times per second. At the same time, most
games have a couple of situations when they behave completely different, the
most prominent example being the case of a three-dimensional shooter or racing
game when the user is in the menu compared to the time when he/she is engaged in
the game itself.

At the implementation level, it would be quite inconvenient to make the same
decision dozens of times per second, specifically the decision of choosing
whether to render the main menu or to compute the world physics and render the
game. The decision tree and, respectively, the chain of flow-control structures
grow as the number of these states that the game might be in, gets bigger and
bigger.

Software development techniques already include a solution to this kind of
situation in the body of a design pattern called the state pattern. \emph{The
Gang of Four}\cite{gof} describe a state object as one that encapsulates some
state-dependent behavior. This maps exactly to what happens in games, various
game states like main menu, active gameplay, cinematic cut-scenes, can be
modeled as objects that have specific implementations of methods for
rendering, performing updates and handling user input every frame.

The Phaser framework makes extensive use of the state pattern and gives
developers the opportunity to model their games as a series of interchanging
states with a set of predefined methods that are called by Phaser at specific
times in the main game loop and can be overridden in order to achieve certain
behavior. Some of the most commonly used methods of the abstract State object
are presented below with a small description of what are they usually used for:


\begin{description}

\item [init()] -- the very first function called when the State starts up;

\item [preload()] -- normally used to load game assets;

\item [create()] -- called when the State is ready to enter the game loop;

\item [update()] -- is for programmer's own use in order to define main game logic;

\item [preRender()] -- called after all Game Objects have been updated, but before any rendering;

\item [render()] -- called after the game is rendered, for final post-processing and style effects;

\item [shutdown()] -- will be called when the State is shutdown.

\end{description}


As the 'Snowfight' game is still quite small at this stage in development, it
features two main game states. Listing \ref{lst:preload_state} shows the code
that defines the 'preload' stage of the game. It is responsible for loading the
assets and bootstrapping all of the game systems like physics and plug-ins. For
this purpose it defines the \emph{preload} and \emph{create} methods.

\lstinputlisting[caption=Preload State, label=lst:preload_state]{preload_state.js}

The 'play' state, on the other hand, represents the description of the main
behavior of the game. It overrides the \emph{create}, \emph{render} and
\emph{update} methods and contains all the logic necessary to control gameplay
process.

\subsubsection{Player's State Machine}

Similarly to the architecture of the whole game, various subsystems can be also
modeled after the state pattern. Specifically the Player's behavior is heavily
dependent on the state that it is in at a given point in time. The need for a
stateful design aggravated at the point of developing the input handling system
as depending on the state of the player, user input had to be processed in very
different ways, for example when a player is disabled, movement controls have no
effect as opposed to the normal activity of the player.

The programming concept behind state handling of the player is similar to the
one used in the game object. A player object holds a reference to a state
manager that is responsible for keeping track of the current state as well as
adding and storing other states. A state object, at the same time, holds the
necessary logic to perform input handling or a player update. It also includes
the definition of the actions that have to be executed when a player enters or
exits a state. With this setup, when the game passes through the game loop and
a player receives a call of the update method, for example, it delegates it to
the state manage which in turn calls the method on the current state object.

Listing \ref{lst:player_states} presents the definition of a
dummy state as well as the code of the state manager.

\lstinputlisting[caption=Player States, label=lst:player_states]{player_states.js}

In his book on game programming patterns\cite{game_patterns}, Robert Nystrom
provides an excellent example of how games can leverage the science behind the
automata theory and how a nested chain of \emph{if} statements can be converted
to an elegant finite state-machine. The Player class in 'Snowfight' tries to
follow that example and model the object as a graph of states and transitions.

Diagram \ref{diag:state_1} from the chapter about system design shows the states
that a player can have and the various transitions that might happen. In some
cases a transition takes place as a result of an event or a condition that
evaluates to truth, at the same time some states transition to the next state
immediately after finishing their job. A good example of such behavior is the
transition from the player's state of \emph{throwing} a snowball back to
\emph{moving}, which happens right away without additional conditions.

\subsubsection{Physics and Rendering}

\subsubsection{Uniform Interface for Input Handling}

The development process becomes increasingly more difficult as the number of
moving parts in the system grows. It also becomes harder to debug and test
individual subsystems in isolated environments and in the case of 'Snowfight'
the development of the game was substantially slowed down by the fact that every
time the page was refreshed, it was necessary to reconnect the controller. In
addition, when some problems appeared it was not clear right away whether the
problem was in the game logic, in the controller code or in the communications
in-between.

One solution to the situation described above was to separate the game from the
controllers and develop it separately by providing the necessary input from the
local keyboard. This way, communication errors and bugs that concern controller
rendering would not stagnate the evolution of the core game.

From the implementation point of view the task required a unified interface for
all possible input sources, in this case only two of them, however, such an
approach opened opportunities to connect an artificial intelligence bot to the
abstract input device, which permitted the addition to the game of non-player
characters (NPC) that would be controlled by the computer.

\subsubsection{High-Level Game Logic}


\subsection{Controller}

\subsubsection{Mouse vs Touch Gestures}

\subsubsection{Trackball Control Element}

\subsubsection{Buttons}


\subsection{Peer to Peer Communication}

\subsubsection{Setting Up a Connection with PeerJS}

\subsubsection{Communication Protocol}

\subsection{Implementation Conclusions}

\clearpage

\cleardoublepage

% Chapter 4
\section{Economic Analysis}
\phantomsection

\subsection{Project Description}

The focus point of this thesis is the development of a toolkit that would
enable software engineers to create with ease web-based applications that
feature real-time communication with mobile devices and leverage all the
sensors and feedback elements that a modern smart-phone can deliver. In the
economic context, however, it is not enough to develop a framework, it is also
crucial to have a demonstration application that implements logic which uses
this framework, so that it can gain public attention and obtain necessary
financial investments to support its further development. The 'Snowfight' game
is a wonderful way to illustrate the power of real-time peer-to-peer
interaction between a desktop browser and a browser of a mobile device. It can
also effectively showcase the majority of the use-cases of the framework.

\subsection{Project Time Schedule}

In order to be successful, a project needs to have an effective development
methodology. Presently, \emph{Agile Software Development} gains popularity and
proves to be a flexible and robust way to create software. This methodology
implies an iterative and adaptive process of development and this in turn
influences the time table that has to be established.

\subsubsection{Objective Determination}

The main objective of the project can be derived from its motivation. The
'Snowfight' should illustrate at its best the features of the toolkit
described in the thesis. The game should represent an isometric multiplayer
arcade in which participants can throw snowballs in each other and score
points. The players should control their characters via their smart-phones,
specifically through a web page rendered in the browser of the phone which
represents a some kind of game controller. The gameplay should feel very
responsive as if the player uses a wired gamepad.

Besides showing off technical features of the framework, the game should
appeal to potential players that would use it as an entertainment asset at
parties or team-building events. It should be engaging and well balanced as to
provide casual, yet challenging, and maybe even addictive, gameplay. The
controller part of the system be supported by most types of mobile devices.

\newpage
\subsubsection{SWOT Analysis}

In order to get a better picture of the product and different paths of its
development it is a great idea to analyze the project from the perspective of
Strengths, Weaknesses, Opportunities and Threats, as it has proven to be an
effective technique in project management.

\paragraph{Strengths}

\begin{itemize}
    \item Intuitive user interface
    \item Multiplayer with up to 10 players
    \item Doesn't require any special programs to be installed
    \item Entertaining
\end{itemize}

\paragraph{Weaknesses}

\begin{itemize}
    \item The technology is available only on mid to high-end devices
    \item The game is somewhat bound by the limitations of the graphical framework
    \item Only one game mode
\end{itemize}

\paragraph{Opportunities}

\begin{itemize}
    \item Add several game modes
    \item Improve graphics art
    \item Use this project as a step-stone for another project
\end{itemize}

\paragraph{Threats}

\begin{itemize}
    \item Lack of visibility on the Internet
    \item Development of similar frameworks by other teams
\end{itemize}

\subsubsection{Time Schedule Establishment}

Most of IT projects' lifetimes consist of 5 basic steps: planning, research,
development, testing and deployment. These steps are in turn subdivided into
smaller parts in order to make the whole process manageable in terms of tasks.
Most of the time table will be allocated to the step of development and
testing as these are the steps with the most workload. Planning and research
by itself shouldn't take up too much time, as requirements usually evolve
during development, also research is a thing that usually never stops. The
development of the project can also be split in tasks that can be performed in
parallel thus minimizing the overall time. Total duration of the project is
computed using the equation \eqref{eq:duration}.

\begin{equation} \label{eq:duration}
D_T = D_F - D_S + T_R,
\end{equation}

\noindent where $D_T$ is the duration, $D_F$ -- the finish date, $D_S$ -- the
start date and $T_R$ -- reserve time. Table \ref{table:schedule} represents
the first iteration of the project schedule. The following notations are used
to improve readability: PM -- project manager, D -- developer, GD -- graphics
designer, SM -- sales manager.

\begin{table}[!h]
\begin{center}
\caption{Time schedule}
\renewcommand{\arraystretch}{1.5}
\begin{tabulary}{1\textwidth}{| C | C | C | C | C |}

\hline \textbf{Nr} & \textbf{Activity Name} & \textbf{Duration (days)} & \textbf{People involved} \\
\hline 1  & Project concept definition          & 5     & PM, GD, SM, D \\
\hline 2  & Market analysis                     & 10    & PM, SM        \\
\hline 3  & Domain analysis                     & 15    & PM, D         \\
\hline 4  & Product requirements specification  & 5     & PM, D         \\
\hline 5  & System modeling (UML)               & 10    & D             \\
\hline 6  & Graphic design                      & 10    & GD            \\
\hline 7  & Framework development               & 20    & D             \\
\hline 8  & Main game development               & 30    & PM, GD, D     \\
\hline 9  & Validation of results               & 10    & PM, GD, D, SM \\
\hline 10 & Documentation                       & 5     & D             \\
\hline 11 & Deployment and testing              & 10    & PM, D         \\
\hline 12 & Active marketing                    & 15    & SM            \\
\hline    & Total time to finish the system     & 145   &               \\
\hline

\end{tabulary}
\label{table:schedule}
\end{center}
\end{table}

\newpage
\subsection{Economic Motivation}

This section aims to give a perspective of the project from the economic point
of view. This includes the expenses and profits encountered as a result of the
project's activity as well as the various strategies of commercializing. All
the costs and prices are given in MDL (Moldavian lei) currency. Tangible and
intangible assets, indirect expenses will also be taken into account. One
important thing to mention is that the game is the fact that it is being
developed as an open-source project and it's primary goal is to popularize the
framework and toolkit underneath. Nevertheless, there are still opportunities
to commercialize the game, especially on the rising wave of the term
'Gamification' which gained substantial popularity lately in various
industries. Thus, the game in modified versions can be licensed to companies
that would like to 'gamify' their internal processes. Another monetization
model that might bring some profit and cover the development expenses is the
so called 'freemium' pricing model. This implies that the game should be
released for free with a limited set of features and the full package
of features would be unlocked to people that purchase a subscription of some
kind.


\newpage
\subsubsection{Asset Expense Evaluation}

In order to compute the potential cost of the product it is necessary to
evaluate the cost of various assets involved in the production of the project.
There are tangible and intangible assets that impose certain costs for the
project, the budget for both types of assets are exposed in tables
\ref{table:tangible_assets} and \ref{table:intangible_assets}.

\begin{table}[!h]
\begin{center}
\caption{Tangible asset expenses}
\renewcommand{\arraystretch}{1.5}
\begin{tabulary}{1\textwidth}{| C | C | C | C | C |}

\hline \textbf{Material} & \textbf{Specification}  & \textbf{Price per unit (MDL)} & \textbf{Quantity} & \textbf{Sum (MDL)}\\
\hline Workstation & Apple Macbook Pro 13' Retina   & 33000  & 2     & \multicolumn{1}{r|}{66000} \\
\hline Smartphone  & Nexus 5 and iPhone 5s          & 9000   & 2     & \multicolumn{1}{r|}{18000}  \\
\hline WebHosting  & Microsoft Azure VM             & 10000  & 1     & \multicolumn{1}{r|}{10000}  \\
\hline \multicolumn{4}{|r|}{Total}                                   & \multicolumn{1}{r|}{96000} \\
\hline
\end{tabulary}
\label{table:tangible_assets}
\end{center}
\vspace{-1.3em}
\end{table}

\begin{table}[!h]
\begin{center}
\caption{Intangible asset expenses}
\renewcommand{\arraystretch}{1.5}
\begin{tabulary}{1\textwidth}{| C | C | C | C | C |}

\hline
\textbf{Material} & \textbf{Specification} & \textbf{Price per unit (MDL)} & \textbf{Quantity} & \textbf{Sum (MDL)} \\
\hline License      & Enterprise Architect Desktop Edition License   & 1900  &   1   & \multicolumn{1}{r|}{1900} \\
\hline License      & Sublime Text 3 License                         & 1500  &   1   & \multicolumn{1}{r|}{1500} \\
\hline License      & Adobe Photoshop CC 2016                        & 2000  &   1   & \multicolumn{1}{r|}{2000} \\
\hline Domain Name  & Domain name registration fee                   & 500   &   1   & \multicolumn{1}{r|}{500} \\
\hline \multicolumn{4}{|r|}{Total}                                                   & \multicolumn{1}{r|}{4900} \\
\hline
\end{tabulary}
\label{table:intangible_assets}
\vspace{-1em}
\end{center}
\end{table}


\newpage
\subsubsection{Direct Expenses}

Besides the costs of assets, the development of the project implies some
direct expenses like consumable materials for the office. Even though the
amounts are  not that big, is important to keep track such expenses as they
tend to add up. These direct expenses are presented in the table
\ref{table:direct_expenses}.

\begin{table}[!h]
\begin{center}
\caption{Direct expenses}
\renewcommand{\arraystretch}{1.5}
\begin{tabulary}{1\textwidth}{| C | C | C | C | C |}

\hline \textbf{Material} & \textbf{Specification} & \textbf{Price per unit (MDL)} & \textbf{Quantity} & \textbf{Sum (MDL)} \\
\hline Whiteboard      & Universal Dry Erase Board & 500    & 1     & \multicolumn{1}{r|}{500} \\
\hline Paper           & A4 - 500 sheet pack       & 60     & 2     & \multicolumn{1}{r|}{120} \\
\hline Marker          & Whiteboard marker         & 15     & 10    & \multicolumn{1}{r|}{150} \\
\hline Pen             & Blue pen                  & 5      & 10    & \multicolumn{1}{r|}{50} \\
\hline \multicolumn{4}{|r|}{Total}                                  & \multicolumn{1}{r|}{820} \\
\hline
\end{tabulary}
\label{table:direct_expenses}
\vspace{-1.5em}
\end{center}
\end{table}


Given the tables above the total amount of expenses in MDL, including tangible
and intangible asset expenses, is the following:

\begin{equation}
T_{e} = 96000 + 4900 + 820 = 11720
\end{equation}


\subsubsection{Salary Expenses}

It is obvious that a major part of project expenses go for the salaries of the
employees. This section describes the computations concerning salary funds.
The starting point is the distribution of per day salaries that the employees
are expected to receive. It looks like this: project manager - 400MDL, sales
manager - 300 MDL, developer - 380 MDL, graphics designer - 350 MDL.

\begin{table}[!ht]
\begin{center}
\caption{Salary expenses}
\renewcommand{\arraystretch}{2}
\begin{tabulary}{1\textwidth}{| C | C | C | C |}

\hline
\textbf{Employee} & \textbf{Work fund (days)} & \textbf{Salary per day (MDL)} & \textbf{Salary fund (MDL)} \\
\hline Project Manager      & 80    & 400   & \multicolumn{1}{r|}{32000} \\
\hline Developer            & 110   & 380   & \multicolumn{1}{r|}{41800} \\
\hline Graphics Designer    & 55    & 350   & \multicolumn{1}{r|}{19250} \\
\hline Sales Manager        & 40    & 300   & \multicolumn{1}{r|}{12000} \\
\hline
\multicolumn{3}{|r|}{Total}                 & \multicolumn{1}{r|}{105050} \\
\hline
\end{tabulary}
\label{table:salaries}
\end{center}
\end{table}

Given the data in the table \ref{table:salaries} it is necessary to compute the amounts of
money that should be payed to social services fund and medical insurance fund.
After this it is possible to sum everything up and obtain the total work
expense amount.

This year the social service fund is approved to be $23\%$, therefore the
salary expenses are computed according to the relation \eqref{eq:fs}.

\begin{equation}\label{eq:fs}
\begin{split}
 FS &= F_{re} \cdot T_{fs} \\
    &= 105050 \cdot 23 \% \\
    &= 24161.50,
\end{split}
\end{equation}

\noindent where $FS$ is the salary expense, $F_{re}$ is the salary expense
fund and $T_{fs}$ is the social service tax approved each year. The medical
insurance fund is computed as

\begin{equation}
\begin{split}
 MI &= F_{re} \cdot T_{mi}\\
    &= 105050 \cdot 4.5\%\\
    &= 4727.25,
 \end{split}
\end{equation}

\noindent where $T_{mi}$ is the mandatory medical insurance tax approved each
year by law of medical insurance and this year it is $4.5\%$.

The total work expense fund can now be computed, given the amounts of social
service and medical insurance taxes.

\begin{equation}
\begin{split}
 WEF &= F_{re} + FS + MI\\
     &= 105050 + 24161.50 + 4727.25\\
     &= 133928.75,
\end{split}
\end{equation}

\noindent where $WEF$ is the work expense fund, $FS$ is the social fund and $MI$
is the medical insurance fund. In that way the total work expense fund was
computed.


\subsubsection{Individual Person Salary}

\subsubsection{Indirect Expenses}

\subsubsection{Wear and Deprecation}

\subsubsection{Product Cost}

\subsubsection{Economic Indicators and Results}

\subsection{Marketing Plan}

\subsection{First Clients}

\subsection{Economic Conclusions}






\clearpage

\cleardoublepage


\phantomsection
\addcontentsline{toc}{section}{Conclusions}
\phantomsection
\addcontentsline{toc}{section}{Conclusions}
\section*{Conclusions}

Now that it can be said that computers are a vital part in human lives, it is
important to keep in mind that this must not be at the expense of other things
like communication and interaction among humans. Technology is often viewed as
having a negative influence on society, but it sure can be used for the opposite
as it depends solely on people themselves what technology can, and what it
cannot do. This thesis represents the result of an effort to boost human
interaction through the use of modern web technology.

'Snowfight' is a web-based multiplayer arcade that employs real-time
communication to enable players to use their mobile phones as game controllers.
It can be used at parties and team-building events in order to entertain and
bring people together in a common activity. The game in itself is a
proof-of-concept application designed to illustrate the use of a toolkit
assembled as the main outcome of the thesis. The obtained toolkit is a
collection of libraries and technologies suitable for this kind of tasks. By the
end of the project the toolkit contained the following items:

\begin{description}
\item [PeerJS] - a communication library which is a wrapper for the WebRTC technology;
\item [HammerJS] - a framework for handling touch screen gestures;
\item [Utility code] - a jQuery plug-in for creating the trackball control element.
\end{description}

During development it could be observed that real-time communication technology
in the web context is still in active development as there where cases when
certain parts of the specification were not supported by some browsers,
moreover, even specifications are still available only as drafts. This results
in a very high probability of breaking changes in future versions of the specs,
that is, a lot of existing code will have to be rewritten in most cases in order
to support new features, and it is likely that the relevant components of the
aforementioned toolkit will be replaced.

At the same time, such a development toolkit has a great potential of evolution,
especially when the specifications concerning RTC technologies will be more or
less standardized. It can be transformed into a full-fledged framework that
would give developers the ability to use in their games a wide range of template
control elements for different purposes and an expressive API for bidirectional
communication between the game and the controller. In the future it could also
have support for motion-detection through the smart-phone's accelerometer and
gyroscope, which would open new opportunities and ideas for game designers.

As for the 'Snowfight' game itself, it could be further refined and improved in
terms of gameplay balance. It would also greatly benefit from a total visual
refactoring as at the moment it uses stock spritesheets and on custom textures
whatsoever. Such changes would appeal to potential users and might stimulate
the development of the framework as well as a set of completely new games.



\clearpage

\cleardoublepage

\cleardoublepage
\addcontentsline{toc}{section}{References}
\printbibliography
\cleardoublepage


\end{document}
