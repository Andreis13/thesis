\section*{Rezumat}

\selectlanguage{romanian}

Această teză reprezintă o încercare de a utiliza tehnologiile web moderne și
anume acelea din domeniul comunicării în timp real, cu scopul de a consolida și
de a încuraja comunicarea și interacțiunea între oameni. Scopul va fi realizat
prin dezvoltarea unui joc multiplayer bazat pe web ce utilizează telefoanele
mobile în calitate de dispozitiv de control, emulînd experiența unei
console de jocuri.

Textul tezei este format din patru capitole, care cuprind etapele individuale a
procesului de dezvoltare a proiectului. Capitolul 1 efectuează o analiză
detaliată a domeniului, oferă un scurt istoric a modului în care a evoluat
interacțiunea omului cu calculatorul și identifică motivația pentru proiectul
tezei. În același capitol sunt prezentate potențialele tehnologii (WebSockets și
WebRTC) ce pot fi utilizate pentru rezolvarea problemei în cauză. Ultimele
secțiuni ale primului capitol conțin o analiză a aplicațiilor care sunt în
prezent pe piață.

Capitolul 2 este axat pe aspectele arhitecturale ale sistemului. El include
rezultatele procesului de modelare în formă de diagrame UML, însoțite de
descrierea sistemului și a părților individuale a acestuia din diferite
perspective. La fel sunt specificate deciziile de proiectare, împreună cu
motivațiile care au stat în spatele lor.

Al treilea capitol oferă o perspectivă detaliată asupra implementării
proiectului. Trei subsecțiuni majore descriu părțile respective ale sistemului,
și anume: jocul, componenta dispozitivului de joc și codul care realizează
comunicarea între acestea. La fel sunt incluse fragmente de cod cu comentarii ce
accentuează tehnicile și instrumentele care au fost folosite în timpul
dezvoltării.

Ultimul capitol realizează o trecere în revistă a proiectului din punct de
vedere economic cu încercarea de a estima potențialul său economic. Acesta
include calcule de diferite tipuri de cheltuieli care în ansamblu constituie
costul total al produsului. Acesta este folosit pentru a calcula indicatorii
economici ai proiectului și prețul de vânzare a unei unități de
produs în cazul în care acesta ar fi introdus pe piață.

La finalul tezei sunt prezentate rezultatele generale a proiectului care au fost
obținute în timpul lucrului cu tehnologiile utilizate în proces de dezvoltare.

\selectlanguage{english}

\clearpage
