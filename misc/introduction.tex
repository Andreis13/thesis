\phantomsection
\addcontentsline{toc}{section}{Introduction}
\section*{Introduction}

Today, people cannot imagine their lives without computers by their side,
although, only seventy years ago, such a world could be the setting of a
bestselling science fiction novel. Back then, it was a research field that had
just planted its roots through the creation of the first ever electronic
computer.

Throughout these years, computers have evolved as much as the ways of
interacting with them. People transitioned from using various switches and knobs
to ergonomic keyboards and touch screens as their input devices. Computers have
also become more connected as with the appearance of the Internet a person could
communicate and interact with a device that is located on the other side of the
planet. So, the present thesis sets its focus specifically on these technologies
viewed inside of a modern context.

Communication and interaction are crucial aspects of human life and technology
can be used to greatly enhance them. Take, for example, team-building events in
a company, that are supposed to bring the employees together for a common
activity like sport or some kind of ice-breaking game. In a situation when the
former might be complicated to organize and the later becomes rather silly when
people get to know each other, computer games come to the rescue. The game
development industry has flourished ever since computers became largely
available, moreover, specialized consoles have been developed to serve the very
purpose of gaming. Game consoles can be a wonderful alternative for team-
building activities, but they require considerable investments and often limit
the number of players that can be engaged in a game at one moment in time.

The thesis proposes a solution to the logistic problems described above. In the
context of recent developments in web technology, and especially the field of
real-time communication (RTC), it is possible to create a web-game based
multiplayer game and web-based controllers that would communicate and emulate
the experience of a game console, with the benefit of being able to run
everywhere where there is a browser installed.

Real-time communication and interaction is a long-running concern of game
developers around the world. As Internet speeds have increased lately,
multiplayer online gaming gained unimaginable levels of popularity and
developers had to come up with solutions and techniques that would make gameplay
as smooth and responsive as possible in order to create the impression that all
players are located in the same room and interact with a single computer through
dedicated input devices. Sadly, this interest in RTC was, until recent,
noticeable only in stand-alone applications and lacked development in the web
context. As the available technologies are very fresh and many specification
documents are still in draft states, this thesis will perform an attempt to
gather and analyze the tools that are currently present on the market and will
show a poof of concept game that is able to cope with the tasks outlined above.

The project proposed in the thesis is a simple web-based multiplayer arcade that
employs real-time communication technologies like WebSockets and WebRTC to
connect and use players' mobile phones as gaming input devices. It aims to bring
an engaging and interactive experience for a group of people located in the same
room without any setup procedures and without a need for specialized gaming gear
like gamepads or joysticks. Various patterns and techniques are expected to be
identified as a result, in order to be used in future projects of this kind.


\clearpage
