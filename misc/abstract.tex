\section*{Abstract}

This thesis represents an attempt to use modern web technology in the field of
real-time communication in order to enhance and encourage communication and
interaction between humans. It is to be done by the development of a web-based
multiplayer game that uses mobile phones as wireless controllers in order to
emulate the experience of a gaming console.

The text of the thesis consists of four chapters that encompass individual steps
of the project's development process. Chapter 1 performs an in-depth analysis of
the domain, provides a brief overview of how did human interaction with
computers evolve, and identifies the motivation for the project. In the same
chapter are depicted the potential technologies (WebSockets and WebRTC) that can
be used for solving the problem in question. The last sections of the first
chapter contain an analysis of applications that are currently present on the
market and follow similar ideas.

Chapter 2 is focused on the architectural aspects of the system. It includes the
results of the modeling process in form of UML diagrams with descriptions of the
system and individual parts of it, from different perspectives. Design decisions
are also specified in this chapter, along with the motivations that stood behind
them.

The third chapter provides a detailed insight of the project implementation. The
three major subsections describe respective parts of the system, specifically:
the main game, the controller component and the code that realizes the
communications between these two. Code excerpts with commentaries try to
emphasize the techniques and tools that were used during development.

The last chapter performs an economic overview of the project with the attempt
to estimate its market potential. It includes calculations of various types of
expenses that result in the total product cost. This is used to compute the
economic indicators of the project and the retail price of the individual unit
of the product if it had to be sold on the market.

At the end of the thesis, the conclusion presents the outcomes of the project
and the general results gain during the work with the technologies used in the
development process.


\clearpage
