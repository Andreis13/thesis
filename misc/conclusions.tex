\phantomsection
\addcontentsline{toc}{section}{Conclusions}
\section*{Conclusions}

Now that it can be said that computers are a vital part in human lives, it is
important to keep in mind that this must not be at the expense of other things
like communication and interaction among humans. Technology is often viewed as
having a negative influence on society, but it sure can be used for the opposite
as it depends solely on people themselves what technology can, and what it
cannot do. This thesis represents the result of an effort to boost human
interaction through the use of modern web technology.

'Snowfight' is a web-based multiplayer arcade that employs real-time
communication to enable players to use their mobile phones as game controllers.
It can be used at parties and team-building events in order to entertain and
bring people together in a common activity. The game in itself is a
proof-of-concept application designed to illustrate the use of a toolkit
assembled as the main outcome of the thesis. The obtained toolkit is a
collection of libraries and technologies suitable for this kind of tasks. By the
end of the project the toolkit contained the following items:

\begin{description}
\item [PeerJS] - a communication library which is a wrapper for the WebRTC technology;
\item [HammerJS] - a framework for handling touch screen gestures;
\item [Utility code] - a jQuery plug-in for creating the trackball control element.
\end{description}

During development it could be observed that real-time communication technology
in the web context is still in active development as there where cases when
certain parts of the specification were not supported by some browsers,
moreover, even specifications are still available only as drafts. This results
in a very high probability of breaking changes in future versions of the specs,
that is, a lot of existing code will have to be rewritten in most cases in order
to support new features, and it is likely that the relevant components of the
aforementioned toolkit will be replaced.

At the same time, such a development toolkit has a great potential of evolution,
especially when the specifications concerning RTC technologies will be more or
less standardized. It can be transformed into a full-fledged framework that
would give developers the ability to use in their games a wide range of template
control elements for different purposes and an expressive API for bidirectional
communication between the game and the controller. In the future it could also
have support for motion-detection through the smart-phone's accelerometer and
gyroscope, which would open new opportunities and ideas for game designers.

As for the 'Snowfight' game itself, it could be further refined and improved in
terms of gameplay balance. It would also greatly benefit from a total visual
refactoring as at the moment it uses stock spritesheets and on custom textures
whatsoever. Such changes would appeal to potential users and might stimulate
the development of the framework as well as a set of completely new games.



\clearpage
